\newpage
\section{Анализ существующих архитектурных решений}
В данной главе рассмотрим существующие архитектуры для выбранных задач классификации и локализации объектов, выберем по три решения для дальнейшего их исследования и сравнения.
\subsection{Задача классификации объекта на изображении}
Классификация объекта сводится к оценке вероятности того, что объект принадлежит к каждому из заданных классов.
\subsubsection{LeNet}
Архитектура, предложенная Яном ЛеКуном в 1998 г. \cite{classifications} (рисунок~\ref{lenet}) для распознавания рукописных цифр (задача MNIST). Сеть принимает на вход изображение $32\times32$ px. Состоит из двух свёрточных слоёв с последующим макспулингом и трёх полносвязных слоёв

\begin{figure}[H]
	\center{\includegraphics[width=1\linewidth]{lenet}}
	\caption{Архитектура LeNet}
	\label{lenet}
\end{figure}

\subsubsection{AlexNet}
Архитектура, предложенная в 2012 г. \cite{classifications} (рисунок~\ref{alexnet}) для решения задачи ImageNet 1000.  Сеть принимает на вход изображение $224\times224\times3$. Состоит из шести свёрточных слоёв и трёх макспулингов, завершается тремя полносвязными слоями. Последовательные свёртки $3\times3$ использовались для экономии памяти в сравнении со свёрткой $5\times5$ или $7\times7$.


\begin{figure}[H]
	\center{\includegraphics[width=0.8\linewidth]{alexnet}}
	\caption{Архитектура AlexNet}
	\label{alexnet}
\end{figure}

\subsubsection{VGG}
Архитектура, предложенная Visual Geometry Group в 2014 г. \cite{classifications} (рисунок~\ref{vgg}). Использовалась для распознавания изображений, отличается большей глубиной, что приводит к необходимости поэтапного обучения (дообучения с добавлением новых слоёв). 

\begin{figure}[H]
	\center{\includegraphics[width=0.7\linewidth]{vgg}}
	\caption{Архитектура VGG}
	\label{vgg}
\end{figure}

\subsubsection{GoogLeNet}
Архитектура, предложенная компанией Google \cite{classifications} (рисунок~\ref{googlenet}). Разрабатывалась для распознавания изображений, отличительной особеннностью является использование ``inception'' блоков и введение вспомогательных функций потерь.

\begin{figure}[H]
	\center{\includegraphics[width=1\linewidth]{googlenet}}
	\caption{Архитектура GoogLeNet}
	\label{googlenet}
\end{figure}

\subsubsection{ResNet}
Архитектура, предложенная компанией Microsoft \cite{classifications} (рисунок~\ref{resnet}), отличающаяся использованием ``Residual'' блоков и отсутствием полносвязных слоёв на выходе (кроме непосредственно одного последнего).
\begin{figure}[H]
	\center{\includegraphics[width=1\linewidth]{resnet}}
	\caption{Архитектуры ResNet}
	\label{resnet}
\end{figure}

\subsubsection{ResNeXt}
Отличие этих архитектур от ResNet заключается в наличии параллельных путей внутри Residual блоков (рисунок~\ref{resnext}).
\begin{figure}[H]
	\center{\includegraphics[width=0.7\linewidth]{resnext}}
	\caption{Отличие Residual блоков архитектуры ResNet (слева) и ResNeXt (справа)}
	\label{resnext}
\end{figure}

\subsubsection{Выбор архитектур для дальнейшего исследования}
Пользуясь проведённым сравнением в статье \cite{classification}, приведём полученную пузырьковую диаграмму (рисунок~\ref{classification}). 

\begin{figure}[H]
	\center{\includegraphics[width=0.8\linewidth]{classification}}
	\caption{Сравнительная диаграмма архитектур, где X-координата является ошибкой Top-1, Y-координата -- это время вывода на GPU в миллисекундах, размер пузырька соответствует размеру модели}
	\label{classification}
\end{figure}

Для дальнейшего исследования выберем ResNet-50, как наиболее сбалансированную модель,  ResNeXt-101-32x8d, как наиболее точную и ResNet-18, как достаточно быструю и лёгкую.

\subsection{Задача локализации объектов на изображении}
Задача локализации -- определить наличие объекта на изображении и сформировать ограничивающую его рамку.
\subsubsection{R-CNN}
Архитектура сети R-CNN (Regions With CNNs) была разработана командой из UC Berkley для применения к задаче локализации \cite{rcnn}. 

Идея сводилась к предобучению свёрточной нейронной сети не на всём изображении целиком, а на предварительно выделенных другим способом регионах, на которых предположительно имеются какие-то объекты. Для выделения регионов использовали  Selective Search. 

В качестве свёрточной сети использовалась готовая архитектура — CaffeNet с заменой последнего классификационного слоя на слой с N+1 выходами (с дополнительным классом для фона).

Несмотря на то, что свёрточная сеть тренировалась на распознавание N+1 классов, в итоге она использовалась только для извлечения фиксированного 4096-размерного вектора признаков. Непосредственным определением объекта на изображении занимались N линейных SVM -- каждый проводил бинарную классификацию по своему типу объектов, определяя есть ли такой в переданном регионе или нет.

Процедуру детектирования объектов сетью R-CNN можно разделить на следующие шаги:
\begin{enumerate} 
\item Выделение регионов-кандидатов при помощи Selective Search.
\item Преобразование региона в размер, принимаемый CNN CaffeNet.
\item Получение при помощи CNN 4096-размерного вектора признаков.
\item Проведение N бинарных классификаций каждого вектора признаков при помощи N линейных SVM.
\item Линейная регрессия параметров рамки региона для более точного охвата объекта 
\end{enumerate}

Процедура иллюстрируется схемой на рисунке~\ref{rcnn}.
\begin{figure}[H]
	\center{\includegraphics[width=0.9\linewidth]{rcnn}}
	\caption{Архитектура R-CNN}
	\label{rcnn}
\end{figure}

\subsubsection{Fast R-CNN}
Авторы Fast R-CNN предложили ускорить процесс за счёт пары модификаций \cite{rcnn}:
\begin{itemize}
	\item Пропускать через CNN не 2000 регионов-кандидатов по отдельности, а всё изображение целиком. Предложенные регионы потом накладываются на полученную общую карту признаков;
	\item Вместо независимого обучения трёх моделей (CNN, SVM, bbox regressor) совместить все процедуры тренировки в одну.
\end{itemize}

Преобразование признаков, попавших в разные регионы, к фиксированному размеру производилось при помощи процедуры RoIPooling. Окно региона делилось на сетку, имеющую фиксированное число ячеек.  По каждой ячейке проводился Max Pooling для выбора только одного значения.

Бинарные SVM не использовались, вместо этого выбранные признаки передавались на полносвязанный слой, а затем на два параллельных слоя: softmax с N+1 выходами (по одному на каждый класс + 1 для фона) и bounding box regressor.

Архитектура показана на рисунке~\ref{fastrcnn}.
\begin{figure}[H]
	\center{\includegraphics[width=0.8\linewidth]{fastrcnn}}
	\caption{Архитектура fast R-CNN}
	\label{fastrcnn}
\end{figure}

\subsubsection{Faster R-CNN}
После улучшений, сделанных в Fast R-CNN, самым узким местом нейросети оказался механизм генерации регионов-кандидатов. В 2015 команда из Microsoft Research смогла сделать этот этап значительно более быстрым \cite{rcnn}. Они предложили вычислять регионы не по изначальному изображению, а опять же по карте признаков, полученных из CNN. Для этого был добавлен модуль под названием Region Proposal Network (RPN). Новая архитектура показана на рисунке~\ref{fasterrcnn}.
\begin{figure}[H]
	\center{\includegraphics[width=0.5\linewidth]{fasterrcnn}}
	\caption{Архитектура faster R-CNN}
	\label{fasterrcnn}
\end{figure}

\subsubsection{YOLO}
You Only Look Once \cite{yolo} (рисунок~\ref{yolo})— это популярная на текущий момент архитектура, которая используется для распознавания множественных объектов на изображении. 
Главная особенность этой архитектуры по сравнению с другими состоит в том, что большинство систем применяют CNN несколько раз к разным регионам изображения, в YOLO CNN применяется один раз ко всему изображению сразу. Сеть делит изображение на своеобразную сетку и предсказывает bounding boxes и вероятности того, что там есть искомый объект для каждого участка. Плюсы данного подхода состоит в том, что сеть смотрит на всё изображение сразу и учитывает контекст при локализации и распознавании объекта. Так же YOLO в 1000 раз быстрее чем R-CNN и около 100x быстрее чем Fast R-CNN.
\begin{figure}[h]
	\center{\includegraphics[width=1\linewidth]{yolo}}
	\caption{Архитектура YOLO}
	\label{yolo}
\end{figure}

\subsubsection{SSD}
Single Shot MultiBox Detector \cite{ssd} -- модель основанная на идее YOLO, то есть одноразового прохода свёртками по изображению, позволяет производить локализацию в реальном времени и по точности близка к Faster R-CNN. Особенностью этой архитектуры является отсутствие полносвязных слоёв и наличие детектирующих блоков на разных уровнях сети. Архитектура показана на рисунке~\ref{ssd}.
\begin{figure}[h]
	\center{\includegraphics[width=1\linewidth]{ssd}}
	\caption{Архитектура SSD}
	\label{ssd}
\end{figure}


\subsubsection{RetinaNet}
Обнаружено, что в одноступенчатом детекторе существует проблема дисбаланса класса переднего плана, и считается, что это главная причина, которая делает производительность одноступенчатых детекторов ниже двухступенчатых. В RetinaNet, одноступенчатом детекторе с использованием комбинации Feature Pyramid Networks + ResNet, и благодаря специальной функции ошибки достигается более высокая точность по сравнению с  Faster R-CNN \cite{retina}. Архитектура показана на рисунке~\ref{retinanet}.
\begin{figure}[h]
	\center{\includegraphics[width=1\linewidth]{retinanet}}
	\caption{Архитектура RetinaNet}
	\label{retinanet}
\end{figure}

\subsubsection{Выбор архитектур для дальнейшего исследования}
Для дальнейшего исследования выберем Faster R-CNN, как лучшего представителя двухступенчатого детектора, и одноступенчатую RetinaNet.